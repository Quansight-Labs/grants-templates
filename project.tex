\sethlcolor{GreenYellow}  
\hl{\textbf Note you need to fill one of these PER PROJECT in your proposal}
\section{Software Project Details: Add name}
\textcolor{gray}{Complete the following table for Open Source Software Project \#1 of your proposal. All URLs should be in
the format https://example.com and only one primary link should be provided.}

\begin{table}[ht]
    \begin{tabular}{@{}ll@{}}
    \toprule
    Software Project Details &    \\ \midrule
    Software project name 
    & -- \\
    Homepage URL 
    & -- \\
    Hosting platform (required) 
    & -- \\
    Main code repository (e.g. GitHub URL) (required)                                      & -- \\
    DOI of major publication(s) describing software project (if applicable)
    & -- \\
    Social media handles (if applicable) 
    & -- \\
    Do you or software project key personnel have\\ commit rights to the code repositories for this\\ software project? (required) 
    & -- \\
    Short description of software project (200 words\\ maximum) (required)\
    & -- \\ \bottomrule
    \end{tabular}
\end{table}

\section{List of known key personnel}
\textcolor{gray}{Key personnel are people involved in the software project who will be supported by the grant if the
application is successful.
Complete the following for the key personnel on the open source software project listed above (up to 5)
(required); enter n/a if any field is not applicable. Personnel to be hired that have not been identified
at this time can be listed in the budget section. You may need to use the scroll bar at the bottom of the
table to scroll right to view and to complete all fields. Alternatively, you can tab to move through and
complete the fields. To add another person/row (up to five), click the box at the end of the row.}

% Per individual:
\begin{enumerate}
    \item First Name
    \item Last Name
    \item email address
    \item Current employer / affiliation
    \item Developer username (e.g. GitHub handle)
    \item Country of Residence
\end{enumerate}

\section{Software Project Metrics: Quality (required):}
\textcolor{gray}{Complete for the open source software project listed above.}

\begin{enumerate}
    \item What is the software project license?
    
    \item What is the main programming language?
    
    \item Does the software project have a code of conduct?
    \subitem \textit{Link (optional; format https://example.com):}
    
    \item Does the software project have end-user documentation?
    \subitem \textit{Link (optional; format https://example.com):}
    
    \item Does the software project have an issue tracker?
    \subitem \textit{Link (optional; format https://example.com):}
    
    \item Does the software project have a community engagement / Q\&A forum (self-hosted, on Stack Exchange etc.)
    \subitem \textit{Link (optional; format https://example.com):}
    
    \item  Does the software project have contribution / coding guidelines?
    \subitem \textit{Link (optional; format https://example.com):}
    
    \item Are there examples or demo notebooks, scripts, and datasets?
    \subitem \textit{Link (optional; format https://example.com):}
    
    \item Is there a corresponding package available in a package manager (PyPi, CRAN, etc.)?
    \subitem \textit{Link (optional; format https://example.com):}
    
    \item Does the software project support continuous integration for testing?
    \subitem \textit{Comment (optional)
}
\end{enumerate}

\section{Software Project Metrics: Impact (optional):}
\textcolor{gray}{Complete the following for the open source software project listed above. Providing metrics is optional
and metrics can be approximate. For each metric, please provide a source, clarify how the metric was
computed, and/or provide any other comments. For monthly metrics, please provide data from the most
recent month for which the corresponding metric is available.}

\begin{enumerate}
\item Complete the following table. List the number and explanation for each, if needed:

% metrics table
\begin{table}[ht]
    \begin{tabular}{lll}
    \hline
    Metric & Number & Comment \\ \hline
    \begin{tabular}[c]{@{}l@{}}Scholarly paper(s) (including\\ preprints) citing or mentioning\\ the software project\end{tabular} 
    & -- 
    & -- \\
    \begin{tabular}[c]{@{}l@{}}Monthly users, if applicable\\ (based on one or more of the\\ following: monthly downloads\\ from websites, monthly\\ downloads from package\\ managers, monthly unique\\ requests for updates, etc.)\end{tabular} 
    & -- 
    & -- \\
    \begin{tabular}[c]{@{}l@{}}Software projects that depend on\\ the project (if applicable)\end{tabular} 
    & -- 
    & -- \\
    \begin{tabular}[c]{@{}l@{}}Monthly visitors to project’s\\ website, discussion forum (e.g.\\ Stack Overflow), or similar\end{tabular} & -- & -- \\ \hline
    \end{tabular}
\end{table}

\item Size of the largest potential user base:

\begin{table}[ht]
    \begin{tabular}{lll}
    \hline
    Metric & Number & Comment \\ \hline
    \begin{tabular}[c]{@{}l@{}}Estimate the potential number of\\ unique users who could adopt\\ this project in the relevant\\ field/discipline. Use as guidance\\ the number of users of\\ comparable projects, the number\\ of papers published in the\\ domain to which the project is\\ applicable, number of labs able\\ to adopt the project, etc.\end{tabular} 
    & -- 
    & -- \\ \hline
    \end{tabular}
\end{table}

\item List of upstream, downstream, or related software projects that the team is contributing to
or receiving contributions from:

\item Additional metrics from project code repositories and package managers:

\textcolor{gray}{Provide a short description of any considerations or caveats we should be aware of when computing
metrics (e.g. a recent change in the name or hosting of the repository), or any additional information you
would like to share about the project’s impact and quality. (maximum of 500 words)}

\end{enumerate}